%% rnaastex.cls is the classfile used for Research Notes. It is derived
%% from aastex61.cls with a few tweaks to allow for the unique format required.
\documentclass[modern]{rnaastex}

\usepackage{graphicx}
\usepackage[suffix=]{epstopdf}
\usepackage{natbib}
\usepackage{amsmath}
\usepackage{xspace}
\usepackage{url}

\newcommand{\ingot}{{\tt ingot}\xspace} % floating space at the end is key!
%\newcommand{\feh}{$[\mathrm{M}/\mathrm{H}]$}
\newcommand{\feh}{$[\mathrm{Fe}/\mathrm{H}]$\xspace}

\begin{document}


\title{Photometric Metallicities for Low-Mass Stars\\ with Gaia and WISE}

%% Note that the corresponding author command and emails has to come
%% before everything else. Also place all the emails in the \email
%% command instead of using multiple \email calls.
\correspondingauthor{James. R. A. Davenport}
\email{jrad@uw.edu}

\author[0000-0002-0637-835X]{James. R. A. Davenport}
\affiliation{Department of Astronomy, University of Washington, Seattle, WA 98195, USA}

\author[0000-0003-3601-3180]{Trevor Z. Dorn-Wallenstein}
\affiliation{Department of Astronomy, University of Washington, Seattle, WA 98195, USA}



%%%%%%%
\section{} 


%Constraining metallicities of main sequence stars across the sky are key to understanding the chemo-dynamical assembly of our Galaxy. 
%Measurements of chemical abundances traditionally require time-intensive spectroscopic observations, often only for individual and bright objects. 
Multi-object spectroscopic surveys --- e.g., the Apache Point Observatory Galactic Evolution Experiment (APOGEE, \citealt{majewski17}) --- have increased the sample of stars with well-measured metallicities by orders of magnitude, but still don't reach brightness limits or sample sizes comparable with photometric surveys. While less precise than spectroscopic measurements, ``photometric metallicities'' from visible and infrared surveys can explore the large scale chemical structure and evolution of our Galaxy \citep[e.g.][]{ivezic08,schmidt16}, though have been largely limited to legacy survey footprints.
%Visible and infrared photometric surveys allow us to develop empirical relations between colors and \feh.
%\citet{ivezic08} developed photometric metallicities using Sloan Digital Sky Survey \citet[SDSS,][]{york00} for 2 million F/G stars. Similarly, \citet{schmidt16} used APOGEE data to derive temperature and metallicity with SDSS and WISE \citep{wise} colors for nearly 4000 K and M dwarfs. However, both studies were naturally limited to the SDSS photometric footprint.

Thanks to the Gaia mission \citep{gaia_dr2}, photometric metallicities are now possible for field stars across the entire sky. Here we present \ingot, a {\it k}--nearest neighbors (KNN) tool to estimate stellar \feh using Gaia and WISE photometry for low-mass stars. We demonstrate this capability by estimating metallicities for three million cool stars, and advocate for more detailed explorations of this technique using WISE and Gaia data.


\begin{figure*}
\centering
\includegraphics[height=2.5in]{color_color}
\includegraphics[height=2.5in]{RZ1}
\caption{
Left: Color--color training diagram for the 35,210 APOGEE stars colored by their measured metallicities. This sample is used to train our {\tt KNN} algorithm \ingot. A reddening vector from \citet{sanders18} for 1 mag of $V$-band extinction is shown for reference.
Right: Average metallicity in our 3 million star ALLWISE-Gaia sample projected on cylindrical  ($R_{xy}$, Z) coordinates. The metal-rich Galactic disk is clearly seen. The apparent break in the disk near the solar position is due to sample incompleteness of high proper-motion objects.
\label{fig:1}}
\end{figure*}

Following Figure 5 of \citet{schmidt16}, we trained \ingot on \feh as a function of ($W1-W2$, $G-J$), shown in Figure \ref{fig:1}. We used 35,210 stars from the APOGEE Stellar Parameters and Chemical Abundances Pipeline \citep[ASCAP,][]{garcia-perez2016}, cross-matched to ALLWISE \citep{mainzer2014} and Gaia DR2 using the CDS X-Match service with a 1 arcsec search radius. The $J$-band magnitude comes from 2MASS \citep{2mass}, and is pre-matched to WISE. $G-J$ (Gaia - 2MASS) spans a wide wavelength range, and is a good proxy for stellar effective temperature. As \citet{schmidt16} note, $W1-W2$ shows a small amplitude gradient that correlates with \feh ($\sim$1 dex in \feh over $\sim$0.1 mag in color).


To represent our data with a flexible model, we used {\tt scikit-learn}'s {\it k}-nearest neighbors (KNN) regression, with the default neighbor distance of $k=5$. This model can rapidly estimate \feh values for any new star given ($W1-W2$, $G-J$), and be easily recomputed given additional axes including the absolute Gaia magnitude ($M_G$). Crude uncertainties can be computed by examining the standard deviation of the measured versus predicted \feh values, which found typical scatter of $\sigma\approx0.11$ dex in our training sample. This does not account for photometric errors, nor correct for extinction in any of the bands.


We applied \ingot to 3.8 million new low-mass stars selected from Gaia and WISE photometry for stars within the color--color box defined in Figure \ref{fig:1}, reaching $\sim$600 pc over the entire sky. The mean \feh for these sources projected into the ($R_{xy}, Z$) plane is shown in Figure \ref{fig:1}. The metal-rich disk of the Milky Way, including the decreasing scale-height with increasing radius, is clearly recovered.


% - original training sample:
% Aok = np.where((Adata[u'FE_H'] > -6))
% Bok = np.where((M_GA >= 3) & 
%               np.isfinite(Adata[u'a_g_val'].values[Aok]) & 
%               (Adata[u'a_g_val'].values[Aok] < 0.5))



% - ADQL:
% SELECT TOP 90000000 gaia.ra, gaia.dec, gaia.parallax, gaia.parallax_error, gaia.a_g_val, gaia.phot_g_mean_mag, gaia.phot_g_mean_flux, gaia.phot_g_mean_flux_error, wise.w1mpro, wise.w1mpro_error, wise.w2mpro, wise.w2mpro_error, tmass.j_m, tmass.j_msigcom FROM gaiadr2.gaia_source as gaia JOIN gaiadr2.allwise_best_neighbour as wise_match ON gaia.source_id = wise_match.source_id JOIN gaiadr1.allwise_original_valid as wise ON wise_match.allwise_oid=wise.allwise_oid JOIN gaiadr2.tmass_best_neighbour as tmass_match ON gaia.source_id = tmass_match.source_id JOIN gaiadr1.tmass_original_valid as tmass ON tmass_match.tmass_oid = tmass.tmass_oid WHERE wise.w2mpro_error < 0.05 AND wise.w1mpro_error < 0.05 AND wise.w2mpro < 14 AND gaia.a_g_val < 0.5 AND gaia.parallax_error/gaia.parallax < 0.1 AND gaia.phot_g_mean_mag + 5*log10(gaia.parallax) - 10 > 4


% - data cuts:
% best = np.where((df['phot_g_mean_mag']-df['j_m'] > 0.5) & 
%                 (df['phot_g_mean_mag']-df['j_m'] < 3) & 
%                 (df['w1mpro']-df['w2mpro'] > -0.13) & 
%                 (df['w1mpro']-df['w2mpro'] < 0.2) & 
%                 (M_G < 11.5) & 
%                 (df['w1mpro_error'] < 0.03) & 
%                 (df['j_msigcom'] < 0.03) & 
%                 (df['parallax_error'] / df['parallax'] < 0.02))


\ingot is publicly available\footnote{https://github.com/jradavenport/ingot}, including the data required to recreate our training sample. We consider \ingot a demonstration of the  potential for simple machine learning tools combined with multi-wavelength data from wide-field surveys to advance galactic chemical cartography. For example, the catalog of over 2 billion point sources from the unWISE project \citep{schlafly2019}, matched to Gaia DR2, is an ideal dataset for such analysis.




%%%%%%%%%
\acknowledgments

%The DIRAC Institute is supported through generous gifts from the Charles and Lisa Simonyi Fund for Arts and Sciences, and the Washington Research Foundation

% https://gea.esac.esa.int/archive/documentation/credits.html
This work has made use of data from the European Space Agency (ESA) mission
{\it Gaia} (\url{https://www.cosmos.esa.int/gaia}), processed by the {\it Gaia}
Data Processing and Analysis Consortium (DPAC,
\url{https://www.cosmos.esa.int/web/gaia/dpac/consortium}). Funding for the DPAC
has been provided by national institutions, in particular the institutions
participating in the {\it Gaia} Multilateral Agreement.

This publication makes use of data products from the Near-Earth Object Wide-field Infrared Survey Explorer (NEOWISE), which is a project of the Jet Propulsion Laboratory/California Institute of Technology. NEOWISE is funded by the National Aeronautics and Space Administration.

\software{Python, IPython \citep{ipython}, NumPy \citep{numpy}, Matplotlib \citep{matplotlib}, SciPy \citep{scipy}, Pandas \citep{pandas}, Astropy \citep{astropy}, Scikit-Learn \citep{scikit-learn}}


\bibliography{bib}


\end{document}
